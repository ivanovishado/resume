%!TEX TS-program = xelatex
%!TEX encoding = UTF-8 Unicode
% Awesome CV LaTeX Template for CV/Resume
%
% This template has been downloaded from:
% https://github.com/posquit0/Awesome-CV
%
% Author:
% Claud D. Park <posquit0.bj@gmail.com>
% http://www.posquit0.com
%
% Template license:
% CC BY-SA 4.0 (https://creativecommons.org/licenses/by-sa/4.0/)
%


%-------------------------------------------------------------------------------
% CONFIGURATIONS
%-------------------------------------------------------------------------------
% A4 paper size by default, use 'letterpaper' for US letter
\documentclass[11pt, a4paper]{awesome-cv}

\usepackage{hyperref}
\hypersetup{
    colorlinks=true,
    linkcolor=blue,
    filecolor=magenta,      
    urlcolor=cyan,
    pdftitle={Overleaf Example},
    pdfpagemode=FullScreen,
    }

% Configure page margins with geometry
\geometry{left=1.4cm, top=.8cm, right=1.4cm, bottom=1.8cm, footskip=.5cm}

% Specify the location of the included fonts
\fontdir[fonts/]

% Color for highlights
% Awesome Colors: awesome-emerald, awesome-skyblue, awesome-red, awesome-pink, awesome-orange
%                 awesome-nephritis, awesome-concrete, awesome-darknight
\colorlet{awesome}{awesome-skyblue}
% Uncomment if you would like to specify your own color
% \definecolor{awesome}{HTML}{CA63A8}

% Colors for text
% Uncomment if you would like to specify your own color
% \definecolor{darktext}{HTML}{414141}
% \definecolor{text}{HTML}{333333}
% \definecolor{graytext}{HTML}{5D5D5D}
% \definecolor{lighttext}{HTML}{999999}

% Set false if you don't want to highlight section with awesome color
\setbool{acvSectionColorHighlight}{true}

% If you would like to change the social information separator from a pipe (|) to something else
\renewcommand{\acvHeaderSocialSep}{\quad\textbar\quad}


%-------------------------------------------------------------------------------
%	PERSONAL INFORMATION
%	Comment any of the lines below if they are not required
%-------------------------------------------------------------------------------
% Available options: circle|rectangle,edge/noedge,left/right
% \photo{./examples/profile.png}
\name{Ivan Fernando}{Galaviz-Mendoza}
\position{Software Engineer}
% \address{42-8, Bangbae-ro 15-gil, Seocho-gu, Seoul, 00681, Rep. of KOREA}

% \mobile{+52 (133) 1234-0239}
\email{galavizivan23@gmail.com}
% \homepage{www.posquit0.com}
\github{ivanovishado}
\linkedin{ivanovishado}
% \gitlab{gitlab-id}
% \stackoverflow{4521547}{ivanovishado}
% \twitter{@twit}
% \skype{skype-id}
% \reddit{reddit-id}
% \medium{ivanovishado}
% \googlescholar{googlescholar-id}{name-to-display}
%% \firstname and \lastname will be used
% \googlescholar{googlescholar-id}{}
% \extrainfo{extra informations}

\quote{"A passionate Software Engineer that loves to go above and beyond."}


%-------------------------------------------------------------------------------
\begin{document}

% Print the header with above personal informations
% Give optional argument to change alignment(C: center, L: left, R: right)
\makecvheader

% Print the footer with 3 arguments(<left>, <center>, <right>)
% Leave any of these blank if they are not needed
\makecvfooter
  {}
  {Ivan Galaviz~~~·~~~Curriculum Vitae}
  {}


%-------------------------------------------------------------------------------
%	CV/RESUME CONTENT
%	Each section is imported separately, open each file in turn to modify content
%-------------------------------------------------------------------------------
%-------------------------------------------------------------------------------
%	SECTION TITLE
%-------------------------------------------------------------------------------
\cvsection{Work Experience}


%-------------------------------------------------------------------------------
%	CONTENT
%-------------------------------------------------------------------------------
\begin{cventries}

%---------------------------------------------------------
  \cventry
    {Senior Software Engineer} % Job title
    {Nubank} % Organization
    {Guadalajara, Mexico} % Location
    {Aug. 2021 - Present} % Date(s)
    {
      \begin{cvitems} % Description(s) of tasks/responsibilities
        \item {Lead the design, implementation, and monitoring of critical projects to enhance the experience of people applying for Nubank's products.}
        \item {Collaborate and coordinate with different stakeholders among several teams to achieve success in our assigned projects.}
        \item {Mentor Mid engineers to empower them to own important tasks and projects.}
        \item {Interview dozens of candidates along all the steps in the hiring pipeline, from reviewing take-home challenges to on-site sessions.}
        \item {Plan, organize, and execute initiatives to increase brand awareness (Building Nu MX Meetups.)}
        \item {Skills: Functional Programming · Datomic · Event Driven Programming · Clojure · Distributed Systems}
      \end{cvitems}
    }

%---------------------------------------------------------
  \cventry
    {Software Engineer III} % Job title
    {Wizeline} % Organization
    {} % Location
    {Mar. 2021 - Jul. 2021} % Date(s)
    {
      \begin{cvitems} % Description(s) of tasks/responsibilities
        \item {Worked on a platform that processed billions of events monthly for another one of the biggest broadcasting and entertainment companies in the US.}
        \item {Continue doing the tasks from my previous role.}
        \item {Skills: Event Driven Programming · Extract, Transform, Load (ETL) · Go · gRPC · Amazon Web Services (AWS)}
      \end{cvitems}
    }

%---------------------------------------------------------
  \cventry
    {Software Engineer II} % Job title
    {} % Organization
    {} % Location
    {Nov. 2019 - Mar. 2021} % Date(s)
    {
      \begin{cvitems} % Description(s) of tasks/responsibilities
        \item {Working on the Authorization/Identity flow for one of the biggest broadcasting companies in the U.S.}
        \item {Interview candidates continuously.}
        \item {Mentor newcomers and junior engineers to help them adopt best practices and develop their soft skills.}
        \item {Plan, organize, and execute initiatives to help community members to experience real interviews in a safe environment.}
        \item {Give courses and talks to the community about programming languages, trying to identify potential engineers and motivate them to start their journey in the IT industry.}
        \item {Contribute to internal development projects to improve different processes.}
        \item {Make continuing efforts to begin my career in Data Engineering/Science.}
        \item {Skills: Amazon Elasticsearch Service · Amazon Dynamodb · Go · Amazon Web Services (AWS) · Mentoring}
      \end{cvitems}
    }
    
%---------------------------------------------------------
  \cventry
    {Software Engineer} % Job title
    {} % Organization
    {} % Location
    {Nov. 2018 - Nov. 2019} % Date(s)
    {
      \begin{cvitems} % Description(s) of tasks/responsibilities
        \item {Developed a chatbot for the biggest real estate company in the U.S.}
        \item {Contributed sporadically to the platform that manages all the chatbots in Wizeline.}
        \item {Gave courses to High School students to implement chatbots easily.}
        \item {Skills: Python · Microsoft Azure · MySQL · Flask}
      \end{cvitems}
    }
    
%---------------------------------------------------------
  \cventry
    {Software Engineer Trainee} % Job title
    {} % Organization
    {} % Location
    {Aug. 2018 - Nov. 2018} % Date(s)
    {
      \begin{cvitems} % Description(s) of tasks/responsibilities
        \item {Worked as a Full-Stack Web Developer in the most used internal project. Developed significant new features and refactored a lot of legacy code.}
        \item {Promoted to a full-time position in two months.}
        \item {Skills: Node.js · React.js · MySQL}
      \end{cvitems}
    }

%---------------------------------------------------------

\end{cventries}

% %-------------------------------------------------------------------------------
%	SECTION TITLE
%-------------------------------------------------------------------------------
\cvsection{Skills}


%-------------------------------------------------------------------------------
%	CONTENT
%-------------------------------------------------------------------------------
\begin{cvskills}

%---------------------------------------------------------
  \cvskill
    {Programming} % Category
    {Python, Go, Clojure, TypeScript, SQL, NoSQL, \LaTeX} % Skills
    
%---------------------------------------------------------
  \cvskill
    {Soft skills} % Category
    {Teamwork, Leadership, Proactiveness, Effective Communication, Mentoring} % Skills
    
%---------------------------------------------------------
  \cvskill
    {Data} % Category
    {Pandas, Information Retrieval, Web Scraping, Streamlit} % Skills

%---------------------------------------------------------
  \cvskill
    {Back-end} % Category
    {Flask, Express, REST API} % Skills

%---------------------------------------------------------
  \cvskill
    {Front-end} % Category
    {React, HTML5, CSS} % Skills
    
%---------------------------------------------------------
  \cvskill
    {Tools} % Category
    {Git} % Skills

%---------------------------------------------------------
  \cvskill
    {DevOps} % Category
    {AWS, Terraform, Vagrant} % Skills

%---------------------------------------------------------
  \cvskill
    {Languages} % Category
    {Spanish, English (IELTS Academic 7.0)} % Skills

%---------------------------------------------------------
\end{cvskills}

%-------------------------------------------------------------------------------
%	SECTION TITLE
%-------------------------------------------------------------------------------
\cvsection{Education}


%-------------------------------------------------------------------------------
%	CONTENT
%-------------------------------------------------------------------------------
\begin{cventries}

%---------------------------------------------------------
  \cventry
    {Bachelor of Computer Engineering} % Degree
    {University of Guadalajara} % Institution
    {Guadalajara, Mexico} % Location
    {Aug. 2013 - Jun. 2018} % Date(s)
    {
      \begin{cvitems} % Description(s) bullet points
        \item {Terminal Project: I made a news scraper for Mexican news websites. It scrapes news daily from around 10 newspapers of each state of Mexico, has a web app that lets a user query for desired news, and classifies news as violent or non-violent, according to their content.}
      \end{cvitems}
    }
    
%---------------------------------------------------------
  \cventry
    {Participant} % Degree
    {Saturdays.AI} % Institution
    {Guadalajara, Mexico} % Location
    {Mar. 2020 - Jun. 2020} % Date(s)
    {
      \begin{cvitems} % Description(s) bullet points
        \item {Participated in an AI Boot Camp where I, alongside a team, developed a project about applying machine learning to predict crimes for bikers in Guadalajara, Mexico.}
      \end{cvitems}
    }
    
%---------------------------------------------------------
  \cventry
    {Summer Research Internship "Programa Delfín"} % Degree
    {National Polytechnic Institute} % Institution
    {Mexico City} % Location
    {Summer 2018} % Date(s)
    {
      \begin{cvitems} % Description(s) bullet points
        \item {Helped during the development of an emotion detector classifier for social networks.}
      \end{cvitems}
    }
    
%---------------------------------------------------------
  \cventry
    {Summer Research Internship "Programa Delfín"} % Degree
    {Research Institute of Applied Mathematics and in Systems, UNAM} % Institution
    {Mexico City} % Location
    {Summer 2017} % Date(s)
    {
      \begin{cvitems} % Description(s) bullet points
        \item {Created a chatbot that helped around 15 people to improve their English and experience with computer programs.}
      \end{cvitems}
    }

%---------------------------------------------------------
\end{cventries}


% \pagebreak

%-------------------------------------------------------------------------------
%	SECTION TITLE
%-------------------------------------------------------------------------------
\cvsection{Extracurricular Activity}


%-------------------------------------------------------------------------------
%	CONTENT
%-------------------------------------------------------------------------------
\begin{cventries}

%---------------------------------------------------------
\cventry
    {Participant} % Degree
    {Saturdays.AI La Paz} % Institution
    {Guadalajara, Mexico} % Location
    {May. 2021 - Jul. 2021} % Date(s)
    {
      \begin{cvitems} % Description(s) bullet points
        \item {Attended Saturdays.AI La Paz Deep Learning edition where we continued developing our project "Violentómetro Online", which aims to detect hate-speech oriented towards women in texts found online.}
      \end{cvitems}
    }

%---------------------------------------------------------
  \cventry
    {Member of "Violentómetro-online Team"} % Affiliation/role
    {DataLab Community} % Organization/group
    {Guadalajara, Mexico} % Location
    {Jun. 2020 - Sep. 2020} % Date(s)
    {
      \begin{cvitems} % Description(s) of experience/contributions/knowledge
        \item {Developed a prototype to detect Spanish hate-speech against women.}
        \item {Won 3rd place against ~20 teams from all around Latin America.}
      \end{cvitems}
    }

%---------------------------------------------------------
  \cventry
    {Member of "Silver Team"} % Affiliation/role
    {Saturdays.AI Guadalajara} % Organization/group
    {Guadalajara, Mexico} % Location
    {Mar. 2020 - Jul. 2020} % Date(s)
    {
      \begin{cvitems} % Description(s) of experience/contributions/knowledge
        \item {Developed a project to estimate the correlation between Guadalajara's crimes and its public bike system.}
        \item {Contributed by always having high-quality code and enabled a distributed team to maintain it effectively.}
        \item {Won 3rd place in an internal contest against seven teams presenting their social-impact machine learning projects.}
      \end{cvitems}
    }

%---------------------------------------------------------
  \cventry
    {Judge} % Affiliation/role
    {Technovation Girls} % Organization/group
    {Guadalajara, Mexico} % Location
    {Apr. 2021 - May 2021} % Date(s)
    {
      \begin{cvitems} % Description(s) of experience/contributions/knowledge
        \item {I served as a judge for six different projects made by girls aged 10-18 years old. These projects aimed to solve a community problem by developing a mobile app.}
      \end{cvitems}
    }
    
%---------------------------------------------------------
  \cventry
    {Mentor} % Affiliation/role
    {} % Organization/group
    {} % Location
    {Feb. 2019 - Jun. 2020} % Date(s)
    {
      \begin{cvitems} % Description(s) of experience/contributions/knowledge
        \item {Successfully mentored two teams, one at the High School level and the other one at the Jr. High School, to develop mobile applications to contribute to their community by solving social problems.}
      \end{cvitems}
    }
    
%---------------------------------------------------------
  \cventry
    {Data Analyst Volunteer} % Affiliation/role
    {University of Guadalajara} % Organization/group
    {Guadalajara, Mexico} % Location
    {Jan. 2018 - Nov. 2018} % Date(s)
    {
      \begin{cvitems} % Description(s) of experience/contributions/knowledge
        \item {Analyzed data collected from students, looking to improve academic courses.}
        \item {Created a system that lets a user recollect data from surveys semi-automatically.}
        \item {Helped with managing courses and delivering homework notes to students.}
      \end{cvitems}
    }

%---------------------------------------------------------
\end{cventries}

%-------------------------------------------------------------------------------
%	SECTION TITLE
%-------------------------------------------------------------------------------
\cvsection{Honors \& Awards}


%-------------------------------------------------------------------------------
%	SUBSECTION TITLE
%-------------------------------------------------------------------------------
\cvsubsection{International}

\begin{cvhonors}

%---------------------------------------------------------
  \cvhonor
    {Honorific Mention} % Award
    {Saturdays.AI La Paz} % Event
    {Guadalajara, MX} % Location
    {2021} % Date(s)

%---------------------------------------------------------
  \cvhonor
    {Winner, 3rd place} % Award
    {Social Data Challenge} % Event
    {Guadalajara, MX} % Location
    {2020} % Date(s)
    
%---------------------------------------------------------
\end{cvhonors}

\cvsubsection{National}

\begin{cvhonors}

%---------------------------------------------------------
  \cvhonor
    {Winner, 3rd place} % Award
    {Saturdays.AI} % Event
    {Guadalajara, MX} % Location
    {2020} % Date(s)

%---------------------------------------------------------
  \cvhonor
    {Award to the Performance Excellence EGEL-Ceneval} % Award
    {CENEVAL} % Event
    {Queretaro, Mexico} % Location
    {2018} % Date(s)

%---------------------------------------------------------
  \cvhonor
    {Research Scholarship} % Award
    {"XXIII Verano de la Investigación Científica y Tecnológica del Pacífico"} % Event
    {Mexico City} % Location
    {2018} % Date(s)
    
%---------------------------------------------------------
  \cvhonor
    {Beneficiary of Program} % Award
    {"Jóvenes de Excelencia"} % Event
    {Mexico} % Location
    {2017} % Date(s)

%---------------------------------------------------------
  \cvhonor
    {Research Scholarship} % Award
    {"XXII Verano de la Investigación Científica y Tecnológica del Pacífico"} % Event
    {Mexico City} % Location
    {2017} % Date(s)

%---------------------------------------------------------
\end{cvhonors}

%-------------------------------------------------------------------------------
%	SECTION TITLE
%-------------------------------------------------------------------------------
\cvsection{Presentation}


%-------------------------------------------------------------------------------
%	CONTENT
%-------------------------------------------------------------------------------
\begin{cventries}

%---------------------------------------------------------
  \cventry
    {Speaker} % Role
    {Go's September 2020 Talk} % Event
    {Guadalajara, Mexico} % Location
    {Sep. 2020} % Date(s)
    {
      \begin{cvitems} % Description(s)
        \item {Talked about concurrency patterns of the Go programming language.}
      \end{cvitems}
    }

%---------------------------------------------------------
  \cventry
    {Speaker} % Role
    {"XXIII Verano de la Investigación Científica y Tecnológica del Pacífico" Congress} % Event
    {Technological University of Bahia de Banderas, Nuevo Vallarta} % Location
    {Aug. 2018} % Date(s)
    {
      \begin{cvitems} % Description(s)
        \item {Described the project I helped with at my 2018 internship "Negative emotion detector for Social Networks."}
      \end{cvitems}
    }

%---------------------------------------------------------
  \cventry
    {Speaker} % Role
    {"XXII Verano de la Investigación Científica y Tecnológica del Pacífico" Congress} % Event
    {Bahia de Banderas Institute of Techonology, Bahía de Banderas} % Location
    {Aug. 2017} % Date(s)
    {
      \begin{cvitems} % Description(s)
        \item {Described the project I helped with at my 2017 internship "Adviser Chatbot for the practice of the English Language."}
      \end{cvitems}
    }

%---------------------------------------------------------
\end{cventries}

%%-------------------------------------------------------------------------------
%	SECTION TITLE
%-------------------------------------------------------------------------------
\cvsection{Writing}


%-------------------------------------------------------------------------------
%	CONTENT
%-------------------------------------------------------------------------------
\begin{cventries}

%---------------------------------------------------------
  \cventry
    {Founder \& Writer} % Role
    {A Guide for Developers in Start-up} % Title
    {Facebook Page} % Location
    {Jan. 2015 - PRESENT} % Date(s)
    {
      \begin{cvitems} % Description(s)
        \item {Drafted daily news for developers in Korea about IT technologies, issues about start-up.}
      \end{cvitems}
    }

%---------------------------------------------------------
  \cventry
    {Undergraduate Student Reporter} % Role
    {AhnLab} % Title
    {S.Korea} % Location
    {Oct. 2012 - Jul. 2013} % Date(s)
    {
      \begin{cvitems} % Description(s)
        \item {Drafted reports about IT trends and Security issues on AhnLab Company magazine.}
      \end{cvitems}
    }

%---------------------------------------------------------
\end{cventries}

%%-------------------------------------------------------------------------------
%	SECTION TITLE
%-------------------------------------------------------------------------------
\cvsection{Program Committees}


%-------------------------------------------------------------------------------
%	CONTENT
%-------------------------------------------------------------------------------
\begin{cvhonors}

%---------------------------------------------------------
  \cvhonor
    {Problem Writer} % Position
    {2016 CODEGATE Hacking Competition World Final} % Committee
    {S.Korea} % Location
    {2016} % Date(s)

%---------------------------------------------------------
  \cvhonor
    {Organizer \& Co-director} % Position
    {1st POSTECH Hackathon} % Committee
    {S.Korea} % Location
    {2013} % Date(s)

%---------------------------------------------------------
  \cvhonor
    {Staff} % Position
    {7th Hacking Camp} % Committee
    {S.Korea} % Location
    {2012} % Date(s)

%---------------------------------------------------------
  \cvhonor
    {Problem Writer} % Position
    {1st Hoseo University Teenager Hacking Competition} % Committee
    {S.Korea} % Location
    {2012} % Date(s)

%---------------------------------------------------------
  \cvhonor
    {Staff \& Problem Writer} % Position
    {JFF(Just for Fun) Hacking Competition} % Committee
    {S.Korea} % Location
    {2012} % Date(s)

%---------------------------------------------------------
\end{cvhonors}



%-------------------------------------------------------------------------------
\end{document}
